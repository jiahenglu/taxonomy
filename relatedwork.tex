\section{Related work} \label{sec:relatedwork}

 Taxonomy is the practice and science of classification. Many taxonomies have a hierarchical structure, but this is not a requirement. In this work, we study to leverage the existing taxonomy to improve the accuracy of string  joins in databases. Such a taxonomy could be constructed manually
through experts and community efforts, as in WordNet [4], Cyc [14], and Freebase. With the advantage of freshness
and informativeness, automatic taxonomy construction has
been extensively studied recently, for example, in [20, 22,
1]. WikiTaxonomy [18] and YAGO [22] may be the most notable efforts, which attempt to derive a taxonomy from Wikipedia categories. In the following, we discuss other related works about similarity functions and similarity join algorithms. 

\cite{journals/vldb/MartinenghiT14} propose a conservative extension of the relational algebra for the taxonomy in which special operators allow the specification of relaxed queries over vaguely structured information.

\cite{conf/sigmod/DingWJHW12}  study how to efficiently process a keyword query whose substitutes are defined by a given taxonomy.

Taxonomies are is-a hierarchies of concepts, topics, or keywords. Since they represent ontology specializations, they may be
used to provide meaningful knowledge representations and, thus, support users in understanding the semantic meaning of a
resource and the related domain.
Several efforts to integrate taxonomies in the Data mining and Knowledge Discovery (KDD) process have been done. For
instance, taxonomies have already been exploited to (i) discover high level correlations among data [9�C13], (ii) perform textual
data analysis and summarization [14,15,35,36], (iii) improve user browsing by looking into the results of Web search engines
[37], and (iv) enhance the quality of recommendation systems [17,38]. This paper focuses on exploiting taxonomy information to
enrich data used for classification. To the best of our knowledge, it is the first attempt to integrate high level and multi-faceted
knowledge into classifier construction.

\noindent \textbf{Similarity functions} In recent years, we have witnessed the emergence of various efforts to enhance the effectiveness of string similarity joins by using synonyms \cite{conf/sigmod/LuLWLW13,conf/icde/ArasuCK08,conf/cpm/BarbayGMR06,conf/vldb/ArvindSR09}.
The traditional similarity functions consider only syntactic similarities, e.g., number of common
words or $q$-grams. But there are many important cases where syntactically different
strings can represent the same real-world object. For example,
``\textsf{Bill}'' is a short form of ``\textsf{William}'', and ``\textsf{ICDE}'' is an abbreviation of  ``\textsf{International Conference on Data Engineering}''.  Given two collections of strings,  synonym-based similarity functions can utilize the available synonyms to find string pairs which are semantically similar. While those methods improve the effectiveness of string joins, a synonym dictionary does not contain information such as
``\textsf{Helsinki is the capital of Finland}'', or `\textsf{`Apple 6 Plus is a new model of iPhone}''. Still, term pairs such as ``\textsf{Helsinki}'' and ``\textsf{Finland}'', ``\textsf{Apple 6 Plus}'' and ``\textsf{iPhone}''  have strong semantic correlations.







\noindent \textbf{Similarity joins}  Several types of Similarity Join have been proposed in the
literature, e.g., distance range join (retrieves all pairs whose distances
are smaller than a predefined threshold $\varepsilon$) [2, 3, 4, 5, 6,
7], k-Distance join (retrieves the k most-similar pairs) [8], and
kNN-join (retrieves, for each tuple in one table, the k nearest neighbors
in another table) [9, 10, 11]. The distance range join
has been one of the most studied and useful types of Similarity
Join. This type of join is commonly referred to simply as Similarity
Join and is the focus of this paper. Among its most relevant
implementation techniques, we find approaches that rely
on the use of pre-built indices, e.g., eD-index [3], D-index [4],
and List of Twin Clusters (LTC) [12]. These techniques strive
to partition the data while clustering together the similar objects.
While these indexing techniques support the SJ operation
they also have some shortcomings: D-index and eD-index may
require rebuilding the index to support queries with different $\varepsilon$,

Approximate string matching includes finding (sub)strings
that resemble a given query string. It is a well-researched
topic and has many applications, such as data cleansing [1],
spelling correction [19], query autocompletion [33], near duplicate
detection [25, 32], approximate named entity recognition
[29], and bioinformatics [20, 27, 35].
Due to the sheer amount of literature in this area, we will
focus on most related recent results and refer readers to the
excellent surveys [13, 22] and tutorials [3, 12, 16] for a comprehensive
treatment of the topic.
Based on the types of the queries, recent work focuses either
on efficient single query processing (typically named string
similarity queries) [1, 6, 9, 11, 18, 25, 28, 31], or the similarity
join which can be treated as processing a batch of similarity
queries [2, 8, 10, 12, 14, 17, 28, 29, 34, 35]. Most recently,
Jiang et al. [15] experimentally evaluate and analyze many of
the existing similarity join algorithms.





