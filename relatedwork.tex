\section{Related work} \label{sec:relatedwork}

Taxonomy is the practice and science of classification. Many taxonomies have a hierarchical structure, but this is not a requirement. In this work, we study to leverage the existing taxonomy to improve the accuracy of string record join in databases. Such a taxonomy could be constructed manually
through experts and community efforts, as in WordNet
[4], Cyc [14], and Freebase. With the advantage of freshness
and informativeness, automatic taxonomy construction has
been extensively studied recently, for example, in [20, 22,
18, 21, 26]. WikiTaxonomy [18] and YAGO [22] may be the
most notable efforts, which attempt to derive a taxonomy
from Wikipedia categories. With more web data, Probase
[26] aims at building a unified taxonomy of worldly facts.


 The contributions of this paper \cite{journals/vldb/MartinenghiT14}   are the following: a simple but solid framework for embedding taxonomies
into relational databases. 




Several types of Similarity Join have been proposed in the
literature, e.g., distance range join (retrieves all pairs whose distances
are smaller than a predefined threshold $\varepsilon$) [2, 3, 4, 5, 6,
7], k-Distance join (retrieves the k most-similar pairs) [8], and
kNN-join (retrieves, for each tuple in one table, the k nearest neighbors
in another table) [9, 10, 11]. The distance range join
has been one of the most studied and useful types of Similarity
Join. This type of join is commonly referred to simply as Similarity
Join and is the focus of this paper. Among its most relevant
implementation techniques, we find approaches that rely
on the use of pre-built indices, e.g., eD-index [3], D-index [4],
and List of Twin Clusters (LTC) [12]. These techniques strive
to partition the data while clustering together the similar objects.
While these indexing techniques support the SJ operation
they also have some shortcomings: D-index and eD-index may
require rebuilding the index to support queries with different $\varepsilon$,

Approximate string matching includes finding (sub)strings
that resemble a given query string. It is a well-researched
topic and has many applications, such as data cleansing [1],
spelling correction [19], query autocompletion [33], near duplicate
detection [25, 32], approximate named entity recognition
[29], and bioinformatics [20, 27, 35].
Due to the sheer amount of literature in this area, we will
focus on most related recent results and refer readers to the
excellent surveys [13, 22] and tutorials [3, 12, 16] for a comprehensive
treatment of the topic.
Based on the types of the queries, recent work focuses either
on efficient single query processing (typically named string
similarity queries) [1, 6, 9, 11, 18, 25, 28, 31], or the similarity
join which can be treated as processing a batch of similarity
queries [2, 8, 10, 12, 14, 17, 28, 29, 34, 35]. Most recently,
Jiang et al. [15] experimentally evaluate and analyze many of
the existing similarity join algorithms.


$\mathbf{Prefix filter.}$ Since the prefix
filter is effective, many methods are proposed to optimize it
for different similarity operators [6,16,20,24,26,28,29]. ED-
join [12,28] proposed a location-based mismatch filter to re-
duce prefix length and a content-based mismatch filter to
reduce the number of candidates for ED. Pivotal prefix filter [4] reduced the prex length for ED. PassJoin [17] pro-
posed segment filter to improve pruning power. PPJoin [29]
used the positions of prefix and suffix to improve pruning
power for token-based similarities. Length filter was pro-
posed to prune dissimilar answers based on length difference [8]. TrieJoin [23] used a different framework that directly computed real similarity using the trie structure.

\smallskip

\noindent \textbf{Bucket histograms.}  The histogram is the most widely used form to store statistical information of the data distributions
in a database. Many different histograms have been proposed in the literature and some have been deployed in commercial RDBMSs. However, almost all previous histograms have one thing in common: they all use buckets to partition the data, although in different ways.
A bucket-based histogram approximates the data in an attribute of a relation by grouping attribute values into buckets and estimating the attribute values and their frequencies based on the summary statistics maintained in each bucket. More specifically, a histogram of an attribute is an approximation of the frequency distribution of the attribute values obtained as follows: The
(attribute value, frequency) pairs of the distribution are partitioned into buckets; The frequency
of each value in a bucket is approximated by the average of the frequencies of all values in the
bucket; The set of values in a bucket is usually approximated in some compact fashion as well.

The popular equi-depth histogram \cite{conf/sigmod/Piatetsky-ShapiroC84}  partitions the interval between the minimum and
maximum attribute value of a column into consecutive subintervals so that the total frequency of
the attribute values for each subinterval is the same. It is interesting to note that these two simple histograms (or their variants) are still used extensively in today��s major commercial RDBMSs like Oracle, Informix, Microsoft SQL Server, Sybase, and Ingres.
